% !TEX TS-program = pdflatex
% !TEX encoding = UTF-8 Unicode

\documentclass[12pt]{article}

\usepackage[utf8]{inputenc}
\usepackage[brazilian]{babel}

%%% PAGE DIMENSIONS
\usepackage{geometry} % to change the page dimensions
\geometry{a4paper} % or letterpaper (US) or a5paper or....
% \geometry{margin=2in} % for example, change the margins to 2 inches all round
% \geometry{landscape} % set up the page for landscape

\usepackage{graphicx} % support the \includegraphics command and options

% \usepackage[parfill]{parskip} % Activate to begin paragraphs with an empty line rather than an indent

%%% PACKAGES
\usepackage{amsfonts}
\usepackage{color}
%\usepackage{booktabs} % for much better looking tables
%\usepackage{array} % for better arrays (eg matrices) in maths
%\usepackage{paralist} % very flexible & customisable lists (eg. enumerate/itemize, etc.)
\usepackage{verbatim} % adds environment for commenting out blocks of text & for better verbatim
%\usepackage{subfig} % make it possible to include more than one captioned figure/table in a single float
% These packages are all incorporated in the memoir class to one degree or another...

\def\Cpp{{C\nolinebreak[4]\hspace{-.05em}\raisebox{.4ex}{\tiny\bf ++}}}
\newcommand{\todo}[1]{\textsf{\color{red}#1}}

%%% END Article customizations

\title{Paralelização do compilador da linguagem de programação Rust}
\author{Yuri Kunde Schlesner}
%\date{} % Activate to display a given date or no date (if empty), otherwise the current date is printed 

\begin{document}
\maketitle

\section{Identificação}

\begin{description}
	\item[Resumo:] \todo{TODO.}
	\item[Período de execução:] Setembro de 2014 a Dezembro de 2014
	\item[Unidades participantes:] ~\\ Curso de Ciência da Computação \\ Departamento de Eletrônica e Computação
	\item[Área de conhecimento:] Ciência da Computação
	\item[Linha de Pesquisa:] Compiladores, Programação Paralela
	\item[Tipo de projeto:] Trabalho de Conclusão de Curso
	\item[Participantes:] ~\\ Profª Andrea Schwertner Charão -- Orientadora \\ Yuri Kunde Schlesner -- Orientando
\end{description}

\section{Introdução}

Um dos principais riscos presentes quando se opera um sistema de computadores é que ele seja atacado por invasores maliciosos. Embora alguns ataques sejam permitidos pela má configuração de servidores e sistemas de segurança, uma significante parte é realizada explorando falhas de programação no software rodando nesses servidores. Erros como \emph{buffer} ou \emph{stack overflows}, acesso a ponteiros nulos e sincronização inadequada entre \emph{threads}, assim como outros tipo de erros de memória, permitem um atacante manipular e acessar informações que não deveria, alterando o funcionamento do software. 

Essas e muitas outras classes de erros só são possíveis devido a escolha de C e \Cpp\ de permitir acessos não-checados a memória. Outras linguagens, como Java, Python e inúmeras outras linguagens de alto-nível, fizeram a escolha de sacrificar performance e controle para eliminar estes tipos de erro, forçando o uso de \emph{garbage collection} e acessos checados a arrays.

No entanto, existem contextos em que o uso dessas técnicas é considerado inaceitável, como por exemplo no desenvolvimento de \emph{kernels} ou \emph{drivers} para sistemas operacionais; em aplicações de tempo real, como automação e controle ou sistemas de prevenção de acidentes; ou aplicações que requerem alta performance, como computação científica, servidores que precisam responder a milhares de requisições por segundo ou jogos. Estes tipos de tarefas geralmente são intituladas ``programação de sistemas''.

A linguagem de programação \emph{Rust}, um projeto de pesquisa da \emph{Mozilla Research}, tem como objetivo preencher este nicho como uma linguagem de programação de sistemas que foca simultaneamente em alta-performance, segurança e expressividade. Ela atinge isso usando um modelo tradicional de compilação prévia (\emph{ahead of time}) e um sistema de tipos que permite a verificação automática dos usos de ponteiros durante a compilação, eliminando a possibilidade de acontecerem erros de memória, mas sem introduzir penalidades excessivas de performance ou consumo de memória.

\section{Objetivos}

\subsection{Objetivo Geral}

O objetivo geral deste trabalho é realizar otimizações e re-estruturar o \emph{rustc}, o compilador oficial de \emph{Rust}, de forma que ele faça melhor uso de processadores com múltiplos núcleos e reutilize o trabalho de compilações anteriores, afim de reduzir o tempo necessário para a compilação de programas.

\subsection{Objetivos Específicos}
\begin{itemize}
	\item Estudar a arquitetura interna do compilador e identificar quais módulos podem realizar seu trabalho independentemente.
	\item Realizar modificações que separem o processo de compilação em partes, e executem estas partes simultaneamente.
	\item Se mostrar-se viável, fazer o compilador detectar etapas para as quais os dados de entrada não mudaram, e portanto não precisam ser re-processadas.
	\item Realizar medições de performance para avaliar os ganhos de tempo atingidos pelas modificações.
	\item Caso estas mostrem-se vantajosas, integrar as mudanças na versão oficial do compilador.
\end{itemize}

\section{Justificativa}

Normalmente, um compilador utilizar apenas um núcleo não é um grande problema, pois cada arquivo fonte é compilado separadamente e portanto basta invocar várias instâncias do compilador simultâneamente para processar cada arquivo. Os objetos intermediários gerados são então passados por um \emph{linker}, que os úne para produzir um binário final, num processo geralmente mais simples, e portanto mais rápido, que a compilação do mesmos.\footnote{O processo de \emph{linking} é, no entanto, também uma etapa demorada em projetos de larga escala com muitos objetos intermediários. Por esse motivo, alguns \emph{linkers} possuem parallelização interna deste processo.}

A linguagem \emph{Rust} diverge deste modelo de compilação e compila um programa ou biblioteca inteira como uma única unidade. Isto é necessário pois não existe uma forma externa das declarações do programa (como arquivos \emph{header} em C), então todo o código fonte precisa ser lido na mesma invocação do compilador para descobrir os tipos e funções presentes. No momento, todo o processo é feito de forma serial.

Para ajudar com a adoção de uma linguagem, é importante que ela seja atrativa para potenciais programadores que estejam buscando uma oportunidade de se livrar de dificuldades que já encontraram em outras linguagens, e rápidos ciclos de iteração, atingidos através duma diminuição do tempo de compilação, são um fator importante a se considerar para atingir esse objetivo. (Longos tempos de compilação são uma crítica frequentemente dirigida a \Cpp, por exemplo.) Portanto, este trabalho ajudará a adoção de \emph{Rust} em projetos maiores que enfrentariam problemas com tempos de compilação.

\section{Revisão de Literatura}

\todo{TODO.}

\section{Metodologia}

Dado seu caráter prático, de resolver um problema específico em um determinado domínio, esta pesquisa se enquadra como pesquisa aplicada. Como o objetivo será de explorar o possível espaço de soluções arquiteturais para o problema, e depois documentar quais destas mostraram-se adequadas, é uma pesquisa exploratória.

\section{Plano de Atividades e Cronograma}

\begin{enumerate}
	\item \label{activity:study} \textbf{Estudar a arquitetura interna do compilador:} Primeiro será feito um levantamento da estrutura geral do compilador, afim de identificar o trabalho necessário para e quais as tarefas que podem ser paralelizadas. Caso fique evidente que serão necessárias grandes mudanças na arquitetura do compilador, estas também serão planejadas com antecedência.
	\item \label{activity:parallelization} \textbf{Divisão/paralelização da compilação:} Aqui será realizado o trabalho de implementação, além da finalização dos detalhes arquiteturais de mais baixo nível, para permitir que o compilador utilize vários núcleos do processador simultâneamente. Este trabalho fará uso do conhecimento adquirido e do planejamento realizado na etapa \ref{activity:study}.
	\item \label{activity:reuse} \textbf{Re-utilização de compilações anteriores:} Será adicionado ao compilador a capacidade de re-utilizar resultados de compilações anteriores nos quais os arquivos fonte que influenciem a compilação não tenham sofrido alterações que afetem o resultado. Aqui será novamente utilizado as informações coletadas na etapa \ref{activity:study}.
	\item \label{activity:benchmark} \textbf{Medições de performance:} Para averiguar o impacto das mudanças efetuadas nas etapas \ref{activity:parallelization} e \ref{activity:reuse}, serão feitas medições de tempo e de performance (utilizando um \emph{profiler}). Esta etapa contribuirá uma medição quantitativa do sucesso obtido nesta pesquisa em alcançar seus objetivos.
	\item \label{activity:merging} \textbf{Integração das mudanças:} Afim de beneficiar a comunidade geral de \emph{Rust}, as modificações efetuadas serão submetidas devolta ao projeto, para aprovação dos membros do time de desenvolvimente e subsequente integração. Espera-se que esse seja um processo contínuo, com pedidos de comentários dos desenvolvedores afim de maximizar a qualidade do trabalho.
\end{enumerate}

Espera-se que o desenvolvimento das atividades siga o seguinte cronograma:

\begin{table}[h]
\centering
\begin{tabular}{c|cccc}
	Etapa & Setembro & Outubro & Novembro & Dezembro \\ \hline
	\ref{activity:study} & \checkmark & & & \\
	\ref{activity:parallelization} & \checkmark & \checkmark & & \\
	\ref{activity:reuse} & & & \checkmark & \checkmark \\
	\ref{activity:benchmark} & \checkmark & \checkmark & \checkmark & \checkmark \\
	\ref{activity:merging} & & \checkmark & & \checkmark \\
\end{tabular}
\caption{Cronograma de Atividades}
\end{table}

\section{Recursos}

Para a realização deste trabalho será utilizado apenas equipamento pessoal do pesquisador, visto que não é necessário o uso de qualquer equipamento especial além de um computador para desenvolvimento.

\section{Resultados Esperados}

Ao término deste trabalho, espera-se que seja disponibilizada uma versão do compilador de \emph{Rust} para realizar a compilação de programas de forma paralela, utilizando vários núcleos de CPU, e re-utilizando trabalho de compilações anteriores, de forma que esta seja completada em menos tempo. Caso os resultados, baseados nas medições de performance realizadas, mostrem-se suficientemente vantajosos, espera-se que as mudanças presentes nessa versão sejam integradas na versão oficial do compilador, de forma a beneficiar todos os membros da comunidade de usuários de \emph{Rust}.

\section{Referências}

\todo{TODO.}

\end{document}
