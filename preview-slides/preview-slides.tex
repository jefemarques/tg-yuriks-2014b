% !TEX TS-program = XeLaTeX
% !TEX encoding = UTF-8 Unicode

\documentclass{beamer}

\mode<presentation>
{
  \usetheme{Dresden}

  \setbeamercovered{transparent}
  \useinnertheme{circles}
  \usecolortheme{lsc}
  \beamertemplatenavigationsymbolsempty
}

\setbeamertemplate{headline}{}
\setbeamercovered{invisible}


\usepackage[brazil]{babel}
% or whatever

%\usepackage[utf8]{inputenc}

%\usefonttheme{serif}

\usepackage[T1]{fontenc}
\usepackage{fontspec}
\setmainfont{Linux Libertine G}
\setsansfont{Source Sans Pro}
\setmonofont{Source Code Pro}

\usepackage{textcomp}
\usepackage{listings}

\lstdefinelanguage{Rust}{
  %keywords={typeof, new, true, false, catch, function, return, null, catch, switch, var, if, in, while, do, else, case, break},
  keywords={struct, fn, let, box, mut, pub, impl, for, match, const},
  %keywordstyle=\color{blue}\bfseries,
  %ndkeywords={class, export, boolean, throw, implements, import, this},
  %ndkeywords={Add, Num},
  %ndkeywordstyle=\color{darkgray}\bfseries,
  %identifierstyle=\color{black},
  sensitive=true,
  comment=[l]{//},
  morecomment=[s]{/*}{*/},
  %commentstyle=\color{purple}\ttfamily,
  stringstyle=\color[rgb]{0.0,0.4,0.65}\ttfamily,
  %morestring=[b]',
  morestring=[b]"
}

\lstdefinestyle{yuriks}{
	basicstyle=\scriptsize\ttfamily,
	commentstyle=\color[rgb]{0,0.5,0}\normalfont,
}

\lstset{
	language=Rust,
	style=yuriks,
	frame=single,
	texcl=true,
	numbers=left,
	showstringspaces=false,
	}
	
\def\Cpp{{C\nolinebreak[4]\raisebox{.2ex}{\scriptsize\bf++}}}
%\def\Cpp{{C\nolinebreak[4]\raisebox{.3ex}{\small++}}}

\title{Exploração da Linguagem Rust para o Desenvolvimento de um \emph{Path Tracer} Paralelo}
\author[Yuri Kunde Schlesner]{Yuri Kunde Schlesner\\Orientador: Profª Drª Andrea Scwertner Charão}
\institute[UFSM]{Ciência da Computação\\Universidade Federal de Santa Maria}
\date{20/10/2014}

\pgfdeclareimage[height=0.75cm]{university-logo}{images/CienciaDaComputacao.png}
\logo{\pgfuseimage{university-logo}}

% Delete this, if you do not want the table of contents to pop up at
% the beginning of each subsection:
\AtBeginSubsection[]
{
  \begin{frame}<beamer>{Outline}
    \tableofcontents[currentsection,currentsubsection]
  \end{frame}
}


% If you wish to uncover everything in a step-wise fashion, uncomment
% the following command: 

%\beamerdefaultoverlayspecification{<+->}


\begin{document}

\begin{frame}
	\titlepage
	%\pgfuseimage{university-logo}
\end{frame}

\begin{frame}{Outline}
  \tableofcontents
\end{frame}

\section{Introdução}
\begin{frame}{Introdução}
	\begin{itemize}
		\item Ferramenta utilizada: Rust
		\begin{itemize}
			\item Princípios e público-alvo
		\end{itemize}
		\item Área de aplicação: Path Tracing
		\begin{itemize}
			\item Requisitos de processamento
		\end{itemize}
	\end{itemize}
\end{frame}

\pgfdeclareimage[height=4.0cm]{smallvcm-output}{images/ggbs_s_vcm.png}
\subsection{Objetivos}
\begin{frame}{Objetivos}
	\begin{itemize}
		\item Portar um renderizador \Cpp\ para Rust
		\begin{itemize}
			\item<2-> SmallVCM
			\item<2-> Pequeno: $\sim$5000 linhas de código
		\end{itemize}
		\item<3-> Paralelização
	\end{itemize}
	\visible<2->{\begin{center}\pgfuseimage{smallvcm-output}\end{center}}
\end{frame}

\subsection{Justificativa}
\begin{frame}{Justificativa}
	\begin{itemize}
		\item Rust quer substituir \Cpp
		\item Desenvolvedores precisam de sugestões e casos de uso
		\item Paralelismo é essencial para extrair performance
	\end{itemize}
\end{frame}

\section{Fundamentação}
\subsection{Rust}
\begin{frame}{Rust}
	\begin{itemize}
		\item Sistema de Regiões
		\begin{itemize}
			\item Gerenciamento de memória sem \emph{garbage collection}.
			\item \emph{Aliasing} e \emph{thread safety}.
		\end{itemize}
		\item \emph{Traits}.
		\item Enums, tuplas e \emph{pattern matching}.
	\end{itemize}
\end{frame}

\begin{frame}[fragile]{Regiões}
	\begin{itemize}
		\item Gerenciamento de memória sem \emph{garbage collection}
	\end{itemize}
	\begin{overprint}
	\onslide<1>
	\begin{lstlisting}
struct Coisa;

fn foobar() -> Coisa {
    let a = Coisa; // Alocado na pilha
    let b = box Coisa; // Alocado no heap
    
    let c = b; // Movido de b para c
    
    a // Move e retorna a
    // Se última linha não termina com \texttt{;}, return é implicito
} // c é liberado. a e b já tinham sido movidos
	\end{lstlisting}

	\onslide<2>
	\begin{lstlisting}
struct Coisa;

fn foobar() -> Box<Coisa> {
    let a = Coisa; // Alocado na pilha
    let b = box Coisa; // Alocado no heap
    
    let c = b; // Movido de b para c
    
    b // Move e tenta retornar b
}
	\end{lstlisting}

	\onslide<3>
	\begin{lstlisting}[numbers=none, breaklines=true]
<anon>:9:5: 9:6 error: use of moved value: `b`
<anon>:9     b // Move e tenta retornar b
             ^
<anon>:7:9: 7:10 note: `b` moved here because it has type `Box<Coisa>`, which is moved by default (use `ref` to override)
<anon>:7     let c = b; // Movido de b para c
	\end{lstlisting}
	\end{overprint}
\end{frame}

\begin{frame}[fragile]{Regiões}
	\begin{itemize}
		\item Previne uso não-sincronizado e invalidação de iteradores.
	\end{itemize}
	
	\begin{overprint}
	\onslide<1>
	\begin{lstlisting}
fn main() {
    let mut n = 52u32;
    
    let a = &n;
    let b = &n; // OK - Dois ponteiros somente-leitura
    
    println!("{} {}", *a, *b);
}
	\end{lstlisting}

	\onslide<2>
	\begin{lstlisting}
fn main() {
    let mut n = 52u32;
    
    let a = &n;
    let b = &n; // OK - Dois ponteiros somente-leitura
    
    *a = 24; // ERRO - Não pode modificar ponteiro compartilhado
    
    println!("{} {}", *a, *b);
}
	\end{lstlisting}
	\begin{lstlisting}[numbers=none, breaklines=true]
<anon>:7:5: 7:12 error: cannot assign to immutable dereference of `&`-pointer `*a`
<anon>:7     *a = 24;
             ^~~~~~~
	\end{lstlisting}
	
	\onslide<3>
	\begin{lstlisting}
fn main() {
    let mut n = 52u32;
    
    let a = &mut n; // OK - Um ponteiro mutável

    
    *a = 24; // OK
    
    println!("{}", *a);
}
	\end{lstlisting}

	\onslide<4>
	\begin{lstlisting}
fn main() {
    let mut n = 52u32;
    
    let a = &mut n;
    let b = &n; // ERRO - n já tem uma referência mutável
    // ...
}
	\end{lstlisting}
	\begin{lstlisting}[numbers=none, breaklines=true]
<anon>:5:14: 5:15 error: cannot borrow `n` as immutable because it is also borrowed as mutable
<anon>:5     let b = &n;
                      ^
<anon>:4:18: 4:19 note: previous borrow of `n` occurs here; the mutable borrow prevents subsequent moves, borrows, or modification of `n` until the borrow ends
<anon>:4     let a = &mut n;
                          ^
	\end{lstlisting}
	\end{overprint}
\end{frame}

\begin{frame}{Traits}
	\begin{itemize}
		\item Similar a \emph{interfaces} de Java, etc.
		\begin{itemize}
			\item Trait contém métodos, mas não campos.
			\item Tipos que implementam um trait podem ser tratados polimórficamente (durante compilação \emph{`e} execução.)
		\end{itemize}
		\item Implementação do trait para o tipo é separado da definição do tipo.
		\begin{itemize}
			\item Novos tipos podem implementar traits existentes.
			\item Novos traits podem vir com implementações para tipos existentes.
			\item Métodos com mesmo nome em traits diferentes podem coexistir no mesmo tipo.
		\end{itemize}
	\end{itemize}
\end{frame}

%\subsection{Path Tracing}
%\begin{frame}{Ray Tracing}
%\end{frame}

\section{Desenvolvimento}
\subsection{Implementação}
\begin{frame}{Visão Geral}
	\begin{itemize}
		\item SmallVCM é dividido entre uma parte comum e algoritmos de renderização específicos que utilizam essa infraestrutura.
		\item A \emph{port} mantém a mesma estrutura de arquivos, tipos e funções, na medida do possível, afim de facilitar comparações.
%		\begin{itemize}
%			\item \texttt{bsdf.hxx}, \texttt{materials.hxx} -- \emph{Shading} de superfície
%			\item \texttt{camera.hxx} -- Câmera virtual
%			\item \texttt{config.hxx} -- Configuração do renderizador e contexto de execução
%			\item \texttt{frame.hxx}, \texttt{ray.hxx}, \texttt{math.hxx} -- Classes de geometria
%			\item \texttt{framebuffer.hxx} -- Armazena resultados de imagem
%			\item \texttt{geometry.hxx} -- Objetos de cena e testes de intersecção
%			\item \texttt{hashgrid.hxx} -- ?
%			\item \texttt{html\_writer.hxx}, \texttt{utils.hxx} -- Funções utilitárias
%			\item \texttt{lights.hxx} -- Tipos de fontes de luz
%			\item \texttt{renderer.hxx} -- Classe abstrata base dos algoritmos de renderização
%			\item \texttt{rng.hxx} -- Gerador de números aleatórios
%			\item \texttt{scene.hxx} -- Gerenciamento de objetos e teste de intersecção
%		\end{itemize}
		\item Alguns módulos tiveram que ser usar uma estratégia diferente para adaptá-los as funcionalidades e limitações de Rust.
	\end{itemize}
\end{frame}

\begin{frame}{Herança}
	\begin{itemize}
		\item Rust não tem herança de \emph{structs}.
		\item Polimorfismo ainda pode ser realizado utilizando \emph{traits}, mas estes não contém dados, apenas métodos.
		\item Requer divisão de classes base em partes, se estas também conterem campos de dados compartilhados.
	\end{itemize}
\end{frame}

\begin{frame}[fragile]{Herança}
	\begin{lstlisting}[language=C++]
class AbstractRenderer {
public:
  AbstractRenderer(const Scene& aScene) {/*...*/}
  virtual ~AbstractRenderer() {}
  virtual void RunIteration(int aIteration) = 0;
  void GetFramebuffer(Framebuffer& oFramebuffer) {/*...*/}
  bool WasUsed() const {/*...*/}
	
  uint mMaxPathLength;
  uint mMinPathLength;
	
protected:
  int mIterations;
  Framebuffer mFramebuffer;
  const Scene& mScene;
}
	\end{lstlisting}
\end{frame}

\begin{frame}[fragile]{Herança}
	\begin{lstlisting}
pub trait AbstractRenderer {
  fn get_base<'a>(&'a mut self) -> &mut RendererBase<'a>;
  fn run_iteration(&mut self, iteration: u32);
}

pub struct RendererBase<'a> {
  pub max_path_length: u32,
  pub min_path_length: u32,

  pub iterations: u32,
  pub framebuffer: Framebuffer,
  pub scene: &'a Scene,
}

impl<'a> RendererBase<'a> {
  pub fn get_framebuffer(&mut self) -> Framebuffer {/*...*/}
  pub fn was_used(&self) -> bool {/*...*/}
}
	\end{lstlisting}
\end{frame}

\begin{frame}[fragile]{Inicialização de Structs}
	\begin{itemize}
		\item Diferentemente de \Cpp, não são permitidos valores não inicializados (sejam variáveis ou campos.)
		\item Alguns construtores unidos com métodos de inicialização:
	\end{itemize}
	\begin{lstlisting}[language=C++]
class Framebuffer {
public:
  Framebuffer() {}

  void Setup(const Vec2F& aResolution) {
    mResolution = aResotluion;
    mResX = int(aResolution.x);
    mResY = int(aResolution.y);
    mColor.resize(mResX * mResY);
    Clear();
  }
  // ...
};
	\end{lstlisting}
\end{frame}
\begin{frame}[fragile]{Inicialização de Structs}
	\begin{lstlisting}
pub struct Framebuffer {
  // ...
}

impl Framebuffer {
  pub fn setup(resolution: Vec2f) -> Framebuffer {
    let res_x = resolution.x as uint;
    let res_y = resolution.y as uint;
    
    Framebuffer {
      color: Vec::from_elem(res_x * res_y, vec3s(0.0)),
      resolution: resolution,
      res_x: res_x,
      res_y: res_y,
    }
  }
}
	\end{lstlisting}
\end{frame}

\begin{frame}[fragile]{Sobrecarga de operadores}
	\begin{itemize}
		\item Sobrecarga de operadores é feita implementando-se \emph{traits} especiais.
		\item \texttt{Add}, \texttt{Sub}, \texttt{Index}, etc.
		\item Bastante verboso comparado com a sintaxe de \Cpp:
		\begin{lstlisting}[language=C++]
friend Vec2x<T> operator+(const Vec2x& a, const Vec2x& b) {
  // ...
}
// ... x8 operadores, x2 tipos de vetor
		\end{lstlisting}
		\begin{lstlisting}
impl<T: Num> Add<Vec2x<T>, Vec2x<T>> for Vec2x<T> {
  #[inline]
  fn add(&self, o: &Vec2x<T>) -> Vec2x<T> {
    // ...
  }
}
// ... x8 operadores, x2 tipos de vetor
		\end{lstlisting}
		\item Pode-se usar macros para diminuir a repetição.
	\end{itemize}
\end{frame}

\begin{frame}[fragile]{Bitflags}
	\begin{itemize}
		\item Macro \texttt{bitflags!} permite definir flags e bitmasks de maneira \emph{type safe}.
	\end{itemize}
	\begin{lstlisting}
bitflags! {
  flags BoxMask: u32 {
    const LIGHT_CEILING    = 1,
    const LIGHT_SUN        = 2,
    const LIGHT_POINT      = 4,
    const LIGHT_BACKGROUND = 8,

    const LARGE_MIRROR_SPHERE = 16,
    const LARGE_GLASS_SPHERE  = 32,
    const SMALL_MIRROR_SPHERE = 64,
    const SMALL_GLASS_SPHERE  = 128,
    const GLOSSY_FLOOR = 256,
    // ...
  }
}
	\end{lstlisting}
\end{frame}
\begin{frame}[fragile]{Bitflags}
	\begin{itemize}
		\item Infelizmente não podem ser usadas como expressões constantes.
		\item Solução: Voltar a utilizar constantes de inteiros, ou usá-las em situações onde não são necessárias expressões constantes.
	\end{itemize}
	\begin{overprint}
	\onslide<1>
	\begin{lstlisting}[language=C++]
uint g_SceneConfigs[] = {
  Scene::kBothSmallSpheres  | Scene::kLightSun,
  Scene::kLargeMirrorSphere | Scene::kLightCeiling,
  Scene::kBothSmallSpheres  | Scene::kLightPoint,
  Scene::kBothSmallSpheres  | Scene::kLightBackground
};
	\end{lstlisting}
	\onslide<2>
	\begin{lstlisting}
fn get_scene_config(scene_id: uint) -> Option<BoxMask> {
 match scene_id {
  1 => Some(scene::BOTH_SMALL_SPHERES | scene::LIGHT_SUN),
  2 => Some(scene::LARGE_MIRROR_SPHERE| scene::LIGHT_CEILING),
  3 => Some(scene::BOTH_SMALL_SPHERES | scene::LIGHT_POINT),
  4 => Some(scene::BOTH_SMALL_SPHERES | scene::LIGHT_BACKGROUND),
  _ => None
 }
}
	\end{lstlisting}
	\end{overprint}
\end{frame}

\begin{frame}[fragile]{Enums}
	\begin{itemize}
		\item Enums não possuem valores inteiros (públicos) associados a elas.
		\item Teste precisa ser feito utilizando \emph{pattern matching}.
	\end{itemize}
	\begin{overprint}
	\onslide<1>
	\begin{lstlisting}[language=C++]
static const char* GetName(Algorithm aAlgorithm)
{
  static const char* algorithmNames[7] =
  {
    "eye light", "path tracing", "light tracing",
    "progressive photon mapping",
    "bidirectional photon mapping",
    "bidirectional path tracing",
    "vertex connection and merging"
  };

  if(aAlgorithm < 0 || aAlgorithm > 7)
    return "unknown algorithm";

  return algorithmNames[aAlgorithm];
}
	\end{lstlisting}
	
	\onslide<2>
	\begin{lstlisting}
fn get_name(self) -> &'static str {
  match self {
    EyeLight => "eye light",
    PathTracing => "path tracing",
    LightTracing => "light tracing",
    ProgressivePhotonMapping => "progressive photon mapping",
    BidirectionalPhotonMapping => "bidirectional photon mapping",
    BidirectionalPathTracing => "bidirectional path tracing",
    VertexConnectionMerging => "vertex connection and merging",
  }
}
	\end{lstlisting}
	\end{overprint}
\end{frame}

\section{Próximos Passos}
\subsection{Terminar a port}
\begin{frame}{Terminar a \emph{port}}
	\begin{itemize}
		\item O SmallVCM continuará sendo portado até um ponto em que esteja funcional e consiga gerar imagens similares ao original.
		\begin{itemize}
			\item Não é necessário portar todos os algoritmos de renderização.
			\item Após a finalização do resto do trabalho, será portado o restante do código, se houver tempo disponível.
		\end{itemize}
		\item Serão feitas comparações de performance com o original, afim de identificar regressões de performance e suas causas.
	\end{itemize}
\end{frame}

\subsection{Paralelização}
\begin{frame}{Paralelização}
	\begin{itemize}
		\item Serão exploradas técnicas de paralelização e como podem ser integradas na versão Rust. Algumas das modalidades a se investigar são:
		\begin{itemize}
			\item Multi-threading
			\item SIMD
			\item GPU Computing
		\end{itemize}
	\end{itemize}
\end{frame}

\begin{frame}{Paralelização}
	\begin{itemize}
		\item Multi-threading
		\begin{itemize}
			\item Trabalho de renderização divisível em sub-regiões menores.
			\item Investigar como o uso de OpenMP do código original pode ser mapeado para a biblioteca padrão de Rust.
		\end{itemize}
	\end{itemize}
\end{frame}

\begin{frame}{Paralelização}
	\begin{itemize}
		\item SIMD
		\begin{itemize}
			\item O \emph{rustc} tem suporte básico a SIMD: Existem tipos que representam um vetor de tamanho fixo, suportando operações aritméticas.
			\item Não existe suporte a operações mais complexas de \emph{shuffling}, comparações ou \emph{masking}.
			\item Investigar como pode ser usado no renderizador e se será necessário adicionar melhor suporte a estas operações no compilador ou biblioteca padrão.
		\end{itemize}
	\end{itemize}
\end{frame}

\begin{frame}{Paralelização}
	\begin{itemize}
		\item GPU Computing
		\begin{itemize}
			\item Existe um projeto \emph{proof of concept} que permite compilar código Rust para CUDA e executá-lo na GPU.
			\item Seria necessário determinar se continua utilizável com novas versões do compilador e se parte do código do renderizador pode ser adaptado para ser acelerado desta maneira.
		\end{itemize}
	\end{itemize}
\end{frame}

\begin{frame}
\titlepage
\end{frame}

\end{document}
