% !TEX TS-program = XeLaTeX
% !TEX encoding = UTF-8 Unicode

\documentclass{beamer}

\mode<presentation>
{
  \usetheme{Dresden}
  \setbeamertemplate{footline}[frame number]

  \setbeamercovered{transparent}
  \useinnertheme{circles}
  \usecolortheme{lsc}
  \beamertemplatenavigationsymbolsempty
}

\setbeamertemplate{headline}{}
\setbeamercovered{invisible}


\usepackage[brazil]{babel}
% or whatever

%\usepackage[utf8]{inputenc}

%\usefonttheme{serif}

\usepackage[T1]{fontenc}
\usepackage{fontspec}
\setmainfont{Linux Libertine G}
\setsansfont{Source Sans Pro}
\setmonofont{Source Code Pro}

\usepackage{textcomp}
\usepackage{siunitx}
\usepackage{listings}

\lstdefinelanguage{Rust}{
  %keywords={typeof, new, true, false, catch, function, return, null, catch, switch, var, if, in, while, do, else, case, break},
  keywords={struct, trait, fn, let, box, mut, pub, impl, for, match, const, if, as, enum, static},
  %keywordstyle=\color{blue}\bfseries,
  %ndkeywords={class, export, boolean, throw, implements, import, this},
  ndkeywords={self, str, u32, uint},
  %ndkeywordstyle=\color{darkgray}\bfseries,
  %identifierstyle=\color{black},
  sensitive=true,
  comment=[l]{//},
  morecomment=[s]{/*}{*/},
  %commentstyle=\color{purple}\ttfamily,
  stringstyle=\color[rgb]{0.0,0.4,0.65}\ttfamily,
  %morestring=[b]',
  morestring=[b]"
}

\lstdefinestyle{yuriks}{
	basicstyle=\scriptsize\ttfamily,
	commentstyle=\color[rgb]{0,0.5,0}\normalfont,
}

\lstset{
	language=Rust,
	style=yuriks,
	frame=single,
	texcl=true,
	numbers=left,
	showstringspaces=false,
	}
	
\def\Cpp{{C\nolinebreak[4]\raisebox{.2ex}{\scriptsize\bf++}}}
%\def\Cpp{{C\nolinebreak[4]\raisebox{.3ex}{\small++}}}

\title{Exploração da Linguagem Rust para o Desenvolvimento de um \emph{Path Tracer} Paralelo}
\author[Yuri Kunde Schlesner]{Yuri Kunde Schlesner\\Orientador: Profª Drª Andrea Scwertner Charão}
\institute[UFSM]{Ciência da Computação\\Universidade Federal de Santa Maria}
\date{20/10/2014}

\pgfdeclareimage[height=0.75cm]{university-logo}{images/CienciaDaComputacao.png}
\logo{\pgfuseimage{university-logo}}

% Delete this, if you do not want the table of contents to pop up at
% the beginning of each subsection:
\AtBeginSubsection[]
{
  \begin{frame}<beamer>{Outline}
    \tableofcontents[currentsection,currentsubsection]
  \end{frame}
}


% If you wish to uncover everything in a step-wise fashion, uncomment
% the following command: 

%\beamerdefaultoverlayspecification{<+->}


\begin{document}

\begin{frame}
	\titlepage
	%\pgfuseimage{university-logo}
\end{frame}

\begin{frame}{Outline}
  \tableofcontents
\end{frame}

\section{Introdução}
\begin{frame}{Introdução}
	\begin{itemize}
		\item Ferramenta utilizada: Rust
		\begin{itemize}
			\item Princípios e público-alvo
		\end{itemize}
		\item Área de aplicação: Path Tracing
		\begin{itemize}
			\item Requisitos de processamento
		\end{itemize}
	\end{itemize}
\end{frame}

\pgfdeclareimage[height=4.0cm]{smallvcm-output}{images/ggbs_s_vcm.png}
\subsection{Objetivos}
\begin{frame}{Objetivos}
	\begin{itemize}
		\item Portar um renderizador \Cpp\ para Rust.
		\item<2-> SmallVCM
		\begin{itemize}	
			\item<2-> Pequeno: $\sim$3200\footnote{Contagem atualizada, mais precisa que a do texto.} linhas de código.
			\item<2-> Paralelização utilizando OpenMP.
			\item<2-> Vários algoritmos permitem modularizar o trabalho.
		\end{itemize}
	\end{itemize}
	\visible<2->{\begin{center}\pgfuseimage{smallvcm-output}\end{center}}
\end{frame}

\subsection{Justificativa}
\begin{frame}[fragile]{Justificativa}
	\begin{itemize}
		\item Rust quer competir com \Cpp
		\item Poucos estudos e aplicações gráficas no momento
		\item Validação da linguagem como alternativa viável
	\end{itemize}
\end{frame}

\section{Fundamentação}
\subsection{Rust}
\begin{frame}{Rust}
%	\begin{itemize}
		Principais diferenciais da linguagem:
		\begin{overprint}
		\onslide<1>
		\begin{itemize}
			\item Compilada, gerando código nativo que não requer \emph{runtime}
			\item Sistema de Regiões/\emph{Lifetimes}
			\begin{itemize}
				\item Gerenciamento de memória sem \emph{garbage collection}
				\item Prevenção de acessos à memória inválida
			\end{itemize}
			\item Prevenção de \emph{aliasing}
			\begin{itemize}
				\item Redução de comportamentos surpreendentes
				\item \emph{Thread safety}
			\end{itemize}
		\end{itemize}
		
		\onslide<2>
		\begin{itemize}
			\item \emph{Traits}.
			\item Tipos de dados algébricos e \emph{pattern matching}.
			\item \texttt{unsafe}: controle quando necessário
		\end{itemize}
		\end{overprint}
%	\end{itemize}
\end{frame}

\begin{frame}[fragile]{\emph{Lifetimes}}
	\begin{itemize}
		\item Controle da duração de validade de variáveis pelo compilador.
		\item Usos da variável depois de sua validade não são permitidas.
		\item Referências tem validade herdada como parte de seu tipo.
		\item Permite receber e retornar referências em funções sem comprometer estas garantias.
	\end{itemize}

	\begin{lstlisting}
fn get_first<T, 'a>(v: &'a Vec<T>) -> &'a T {
    return v[0];
}
	\end{lstlisting}
\end{frame}

\begin{frame}{Prevenção de \emph{aliasing}}
	\begin{itemize}
		\item Dois tipos de referências: \texttt{\&T} e \texttt{\&mut T}.
		\item Referências compartilhadas não podem ser usadas para modificar objetos.
		\item Referencias mutáveis não podem ser compartilhadas.
		\item Previne que escritas em uma referência sejam visíveis através de outras.
		\item Automaticamente previne condições de corrida em código \emph{multi-threaded}.
	\end{itemize}
\end{frame}

\begin{frame}{Traits}
	\begin{itemize}
		\item Similares à \emph{interfaces} ou \emph{type classes}.
		\begin{itemize}
			\item Trait contém métodos, mas não campos.
			\item Tipos que implementam um trait podem ser tratados polimórficamente (durante compilação e execução.)
		\end{itemize}
		\item Implementação do trait para o tipo é separado da definição do tipo.
		\begin{itemize}
			\item Novos tipos podem implementar traits existentes.
			\item Novos traits podem vir com implementações para tipos existentes.
			\item Métodos com mesmo nome em traits diferentes podem coexistir no mesmo tipo.
		\end{itemize}
	\end{itemize}
\end{frame}

\begin{frame}{\texttt{unsafe}}
	\begin{itemize}
		\item Permite burlar as regras do sistema de tipos.
		\item Para implementação de código de baixo-nível ou estruturas de dados.
		\item Não altera semântica existente, apenas permite mais tipos de operações.
		\item Blocos de código facilmente identificáveis e auditáveis.
	\end{itemize}
\end{frame}

\section{Desenvolvimento}
\subsection{Implementação}
\begin{frame}{Visão Geral}
	\begin{itemize}
		\item O SmallVCM é dividido em uma parte comum e em algoritmos de renderização específicos que utilizam esta infraestrutura.
		\item A \emph{port} mantém a mesma estrutura de arquivos, tipos e funções, na medida do possível, afim de facilitar comparações.
		\item Alguns módulos precisaram de uma estratégia diferente para adaptá-los as funcionalidades e limitações de Rust.
	\end{itemize}
\end{frame}

\begin{frame}{Herança}
	\begin{itemize}
		\item Rust não tem herança de \emph{structs}.
		\item Polimorfismo ainda pode ser realizado utilizando \emph{traits}, mas estes não contém dados, apenas métodos.
		\item Requer divisão de classes base em partes, se estas também conterem campos de dados compartilhados.
	\end{itemize}
\end{frame}

\begin{frame}{Herança}
	\begin{itemize}
		\item Solução: separar dados de métodos.
		\item Criar estrutura para conter os dados da classe base e disponibilizá-la através do \emph{trait}.
		\item Mais verboso e menos flexível.
		\item Na prática, em arquiteturas feitas do zero, geralmente se evita este problema.
	\end{itemize}
\end{frame}

\begin{frame}[fragile]{Inicialização de Structs}
	\begin{itemize}
		\item Diferentemente de \Cpp, não são permitidos valores não inicializados ou ponteiros nulo
		\item Alguns construtores unidos com métodos de inicialização.
		\item Maioria dos ponteiros pôde ser transformado em referências não-nulas.
		\item Em alguns fez-se uso de \texttt{Option<T>}, removendo nulos implícitos.
	\end{itemize}
\end{frame}

\begin{frame}[fragile]{Sobrecarga de operadores}
	\begin{itemize}
		\item Sobrecarga de operadores é feita implementando-se \emph{traits} especiais.
		\item \texttt{Add}, \texttt{Sub}, \texttt{Index}, etc.
		\item Bastante verboso comparado com a sintaxe de \Cpp:
		\begin{lstlisting}[language=C++]
friend Vec2x<T> operator+(const Vec2x& a, const Vec2x& b) {
  // ...
}
// ... x8 operadores, x2 tipos de vetor
		\end{lstlisting}
		\begin{lstlisting}
impl<T: Num> Add<Vec2x<T>, Vec2x<T>> for Vec2x<T> {
  #[inline]
  fn add(&self, o: &Vec2x<T>) -> Vec2x<T> {
    // ...
  }
}
// ... x8 operadores, x2 tipos de vetor
		\end{lstlisting}
		\item Pode-se usar macros para diminuir a repetição.
	\end{itemize}
\end{frame}

\begin{frame}[fragile]{Enums}
	\begin{itemize}
		\item Enums não possuem valores inteiros (públicos) associados a elas.
		\item Teste precisa ser feito utilizando \emph{pattern matching}.
	\end{itemize}

	\begin{lstlisting}
fn get_name(self) -> &'static str {
  match self {
    EyeLight => "eye light",
    PathTracing => "path tracing",
    LightTracing => "light tracing",
    ProgressivePhotonMapping => "progressive photon mapping",
    BidirectionalPhotonMapping => "bidirectional photon mapping",
    BidirectionalPathTracing => "bidirectional path tracing",
    VertexConnectionMerging => "vertex connection and merging",
  }
}
	\end{lstlisting}
\end{frame}

\begin{frame}{Instabilidade da Linguagem}
	\begin{itemize}
		\item Em ativo desenvolvimento. Ainda não existem versões estáveis.
		\item Vários \emph{commits} diariamente. Semântica da linguagem era alterada quase que semanalmente.
		\item Necessitou de constantes re-ajustes ao código para se atualizar.
		\item Versão estável 1.0 prevista para fim do ano ou início de 2015.
	\end{itemize}
\end{frame}

\section{Resultados e Conclusão}
\subsection{Resultados}
\begin{frame}{Resultados}
	\begin{itemize}
		\item Foi portado infraestrutura e um algoritmo do SmallVCM, representando aproximadamente metade do código original.
		\item Renderizador portado \texttt{EyeLight} produz resultados fiéis ao original.
		\item Paralelizado utilizando a biblioteca ``rayon'', que oferece uma interface simples para computações \emph{multi-threaded fork/join}.
		\item Infelizmente, outros algoritmos não foram portados.
	\end{itemize}
\end{frame}

\begin{frame}{Desempenho}
	\begin{itemize}
		\item Versão em Rust obteve resultados acima do esperado.
		\item Testes foram feitos com gcc e clang, que assim como o rustc também utiliza LLVM.
		\item Cerca de 7\% mais rápido que o clang, utilizando mesmos algoritmos e arquitetura de código similar.
	\end{itemize}

	{\small
	\begin{tabular}{|c|c|c|c|c|} \hline
	Iterações & \emph{Threads} & rustc & clang & gcc \\ \hline
	100 & 1 & \SI{7.665}{\s} (85.45\%) & \SI{8.253}{\s} (92\%) & \SI{ 8.970}{\s} (100\%) \\
	400 & 4 & \SI{8.802}{\s} (84.59\%) & ---\footnote{clang não possui suporte a OpenMP.} & \SI{10.405}{\s} (100\%) \\ \hline
	\end{tabular}
	}
\end{frame}

\subsection{Conclusão}
\begin{frame}{Conclusão}
	\begin{itemize}
		\item Objetivo de portar parte do SmallVCM, mantendo estrutura similar, para a linguagem Rust.
		\item Simplicidade do código foi afetada negativamente pelo objetivo de manter a mesma estrutura, devido a diferenças nas linguagens.
		\item \emph{Port} de um algoritmo com sucesso, obtendo resultados corretos.
		\item Aumento de desempenho quando comparado ao código original, mesmo sem realização de otimizações específicas.
	\end{itemize}
\end{frame}

\begin{frame}{Trabalhos Futuros}
%	\begin{itemize}
		Possíveis continuações para o trabalho:
		\begin{itemize}
			\item Portar mais algoritmos, a fim de verificar se a vantagem de desempenho mantém-se neles.
			\item Realizar uma nova \emph{port} ou re-estruturar a \emph{port} atual utilizando uma estrutura mais idiomática e adequada a Rust e observar ganhos ou perdas em clareza e performance.
		\end{itemize}
%	\end{itemize}
\end{frame}

\begin{frame}
\titlepage
\end{frame}

\end{document}
